\documentclass[12pt,a4paper,titlepage]{article}

\usepackage[margin=2.5cm]{geometry}
\usepackage[latin1]{inputenc}
\usepackage{amsmath}
\usepackage{amsfonts}
\usepackage{amssymb}
\usepackage{graphicx}
\usepackage{listings}
\usepackage{datetime}
\usepackage{multirow}
\usepackage{enumitem}
\usepackage[none]{hyphenat}
\usepackage{hyperref}

\hypersetup{
    colorlinks,
    citecolor=black,
    filecolor=black,
    linkcolor=black,
    urlcolor=black
}

\newcommand{\HRule}{\rule{\linewidth}{0.5mm}}

\setlist{noitemsep}
\setlist[itemize]{topsep=0pt}


\begin{document}

\begin{titlepage} 
\begin{center}

\includegraphics[width=1\textwidth]{./logos/freeems}\\[1cm] \vfill

\HRule \\[0.8cm]
{ \huge \bfseries Memory Management Scheme}\\[0.4cm]
\HRule \vfill

\Large 
%\emph{Author:}\\
%\textsc{Fred Cooke} \vfill

{\large Version 0.2 --- 27th November 2008}

\end{center}
\end{titlepage} 

\tableofcontents
\thispagestyle{empty}
\newpage


\setlength{\parindent}{0.0in}
\parskip 12pt


\section{Purpose / Preamble} 

This document exists to summarise and explain the memory available on the
\texttt{XDP512} processor and how it is being used in the FreeEMS application
firmware.

This document is believed to be true and correct at the time of writing.
Obviously the firmware is under continuous active development and there will
be changes to the way things are done as time passes. The code is the best
reference, however I will endeavour to keep this up to date.




\section{Overall Memory Layout}

The \texttt{XDP512} has a 16 bit data bus and main addressing scheme. This
limits addressable space to a total of 64k (65536 bytes). The 64k is made up of
the following parts:

\begin{itemize}
\item 2k of control registers from \texttt{0x0000-0x07FF} with length \texttt{0x0800}
\item 2k of addressable EEPROM from \texttt{0x0800-0x0FFF} with length \texttt{0x0800}
\item 12k of addressable RAM from \texttt{0x1000-0x3FFF} with length \texttt{0x3000}
\item 48k of addressable Flash from \texttt{0x4000-0xFFFF} with length \texttt{0xC000}
\end{itemize}

The registers are permanently exposed and linearly addressed, and therefore
accessed directly at any time. The EEPROM, RAM and Flash, however, all use a
paging scheme to expose a larger space than would otherwise be possible. The
total amount of each accessible using paging is as follows:

\begin{itemize}
\item 4k of EEPROM
\item 32k of RAM
\item 512k of Flash
\end{itemize}




\section{EEPROM Memory Layout}

EEPROM is not used in FreeEMS at this time, however the 2k is made up of the
following parts:

\begin{itemize}
\item 1k of EEPROM page window from \texttt{0x0C00-0x0FFF} with length \texttt{0x0400}
\item 1k of direct addressable EEPROM from \texttt{0x0800-0x0BFF} with length \texttt{0x0400}
\end{itemize}

The following are valid \texttt{EPAGE} values: \texttt{0xFF}, \texttt{0xFE},
\texttt{0xFD}, \texttt{0xFC}. The fixed 1K page from \texttt{0x0C00} to
\texttt{0x0FFF} is the page exposed by setting \texttt{EPAGE-0xFF}.
Doing that is pointless however as you then have two copies of the same EEPROM
exposed at the same time. The reset value of \texttt{EPAGE} is \texttt{0xFE}
and ensures that there is a linear EEPROM space available between addresses
\texttt{0x0800} and \texttt{0x0FFF} out of reset. Therefore, the useful set of
\texttt{EPAGE} values are: \texttt{0xFE}, \texttt{0xFD}, \texttt{0xFC}




\section{RAM Memory Layout}

This section details how the available RAM is laid out and used in FreeEMS.
The 12k of direct addressable RAM consists of the following parts:

\begin{itemize}
\item 4k of RAM page window from \texttt{0x1000-0x1FFF} of length \texttt{0x1000}
\item 8k of direct addressable RAM from \texttt{0x2000-0x3FFF} of length \texttt{0x2000}
\end{itemize}

The following are valid \texttt{RPAGE} values: \texttt{0xFF, 0xFE, 0xFD, 0xFC,
0xFB, 0xFA, 0xF9, 0xF8}. The fixed 4K page from \texttt{0x2000-0x2FFF} is the
page exposed by setting \texttt{RPAGE} to \texttt{0xFE}. The fixed 4K page
from \texttt{0x3000-0x3FFF} is the page exposed by setting \texttt{RPAGE} to
\texttt{0xFF}. As with the EEPROM, doing that is pointless because you are just
exposing the region twice at the same time. The reset value of \texttt{RPAGE}
is \texttt{0xFD} and ensures that there is a linear RAM space available between
addresses \texttt{0x1000} and \texttt{0x3FFF} out of reset. Therefore the
useful set of \texttt{RPAGE} values is: \texttt{0xFD, 0xFC, 0xFB, 0xFA, 0xF9,
0xF8}.

In FreeEMS we use the 6 available pages as two triplets for tune switching
(like table switching, but almost everything is switched). At any given time
the firmware is using 3 pages that contain one copy of the following:

\begin{itemize}
\item 3 VE Tables
\item 1 Lambda Target Table
\item 2 Ignition Timing Tables
\item 2 Injection Timing Tables
\item 4 blocks of tunable items including all small tables etc.
\end{itemize}

Each of these 12 items is 1k in size totalling 12k which doubled (for tune
switching) is 24k. These are available in 3 sets of four blocks at a time.
VE and Lambda are together, Timing tables are together and the tunable items
are together. The remaining 8k is directly addressable linear space, used as
follows:

\begin{itemize}
\item 2k buffer for reception of communications data
\item 2k buffer for transmission of communications data
\item 4k of general purpose RAM
\end{itemize}

The 4k of general purpose RAM is used in the traditional way with all of the
global variables at the low end and the stack at the high end growing down
toward the globals. Currently there are approximately 1k of global variables in
that region.



\section{Flash Memory Layout}

This section details how the available Flash is laid out and used in FreeEMS.
The 48k of direct addressable Flash consists of the following parts:

\begin{itemize}
\item 16k of direct addressable Flash from \texttt{0x4000-0x7FFF} of length \texttt{0x4000}
\item 16k of Flash page window from \texttt{0x8000-0xBFFF} of length \texttt{0x4000}
\item 16k of direct addressable Flash from \texttt{0xC000-0xFFFF} of length \texttt{0x4000}
\end{itemize}

The following are valid \texttt{PPAGE} values: \texttt{0xFF, 0xFE, 0xFD, 0xFC,
0xFB, 0xFA, 0xF9, 0xF8, 0xF7, 0xF6, 0xF5, 0xF4, 0xF3, 0xF2, 0xF1, 0xF0, 0xEF,
0xEE, 0xED, 0xEC, 0xEB, 0xEA, 0xE9, 0xE8, 0xE7, 0xE6, 0xE5, 0xE4, 0xE3, 0xE2,
0xE1, 0xE0}.

The fixed 16K page from \texttt{0x4000-0x7FFF} is the page exposed by setting
\texttt{PPAGE} to \texttt{0xFD}. The fixed 16K page from \texttt{0xC000-0xFFFF}
is the page exposed by setting \texttt{PPAGE} to \texttt{0xFF}. As with the RAM
and EEPROM, doing that is pointless because you are just exposing the region
twice at the same time.

The reset value of \texttt{PPAGE} is \texttt{0xFE} and ensures that there
is a linear Flash space available between addresses \texttt{0x8000} and
\texttt{0xBFFF} out of reset. Therefore, the useful set of \texttt{PPAGE}
values are: \texttt{0xFE, 0xFC, 0xFB, 0xFA, 0xF9, 0xF8, 0xF7, 0xF6, 0xF5, 0xF4,
0xF3, 0xF2, 0xF1, 0xF0, 0xEF, 0xEE, 0xED, 0xEC, 0xEB, 0xEA, 0xE9, 0xE8, 0xE7,
0xE6, 0xE5, 0xE4, 0xE3, 0xE2, 0xE1, 0xE0}.

The 512k of available flash space is made up of four 128k blocks, each with
eight 16k pages. Currently, due to limitations of hcs12mem and/or the serial
monitor, we can only use up to eight 16k pages totalling 128k of Flash. Two of
those pages are permanently exposed in linear space (\texttt{0xFF/text} and
\texttt{0xFD/text1}) leaving six pages of Flash for dynamic use in the short to
medium term.

The two linear regions are called \texttt{text} and \texttt{text1}, the six Flash pages are
swapped in and out automatically by the compiler whenever needed. In them we
store the following:

\begin{itemize}
\item Commonly used items and the serial monitor
\item All Interrupt Service Routines
\item Lookups and functions operating on them
\item Fuel tables and functions operating on them
\item Timing tables and functions operating on them
\item Tunable items and functions operating on them
\item General function storage area one
\item General function storage area two
\end{itemize}




\section{Where Things Are}

The current break down of usage within those pages is as follows:

Commonly used items and the serial monitor:

\begin{itemize}
\item All functions without a page specified are stored in text region by default
\item The serial monitor takes up 2k of space at the high end of text
\end{itemize}

All Interrupt Service Routines:

\begin{itemize}
\item All placed in text1 region as a way of tracking flash usage
\end{itemize}


Lookups and functions:

\begin{itemize}
\item 2k CHT lookup table
\item 2k IAT lookup table
\item 2k MAF lookup table
\item Core variable generation function
\end{itemize}


Fuel tables and functions:

\begin{itemize}
\item 6 1k VE tables
\item 2 1k Lambda tables
\item Copy flash to ram function
\end{itemize}


Timing tables and functions:

\begin{itemize}
\item 4 1k Ignition timing tables
\item 4 1k Injection timing tables
\end{itemize}


Tunable items and functions:

\begin{itemize}
\item 8 1k blocks of tunable items
\end{itemize}


General paged function space one:

\begin{itemize}
\item Assorted functions and data used purely within them
\end{itemize}


General paged function space two:

\begin{itemize}
\item Assorted functions and data used purely within them
\end{itemize}


Note: The ISR code \emph{must} be kept in linear space such that they can be
called asynchronously at any time.




\section{How They Get There}

During compilation directives are used to locate various pieces of data and code into specific sections. These sections are defined in the memory.x file which lays out the sizes and offsets from the codes point of view. Regions labeled as certain sections are directed and redirected into specific sections based on the contents of the linker script and also regions.x for FreeEMS specific regions.

The linker as instructed by the linker script then moves and splits and merges these sections to form final sections to be included in the output.

Finally, the object copy program takes these blocks and further relocates them into a loadable image according to rules passed on the command line in the Makefile.

There are many sections that come from the compilation process, however we only use a subset of those.

The following Flash sections are present in the final loadable image :


\begin{itemize}
\item rodata - Constants as declared with the const keyword.
\item data - Static initialisation to non zero values.
\item text - Code that must be present in linear space to function correctly.
\item text1 - ISR code that must be present in linear space to be called asynchronously.
\item vectors - The jump table of void function addresses for all interrupt handlers.
\item fixedconf1 - The first block of non live tunable configuration.
\item fixedconf2 - The second block of non live tunable configuration.
\item ppageF8 - General purpose flash page.
\item ppageF9 - General purpose flash page.
\item dpageFA - Non-volatile storage of the fuel tables.
\item fpageFA - Function to load the fuel tables.
\item dpageFB1 - dpageFB8 - Non-volatile storage of the tunable tables.
\item fpageFB - Function to load the tunable tables.
\item dpageFC - Non-volatile storage of the timing tables.
\item fpageFC - Function to load the timing tables.
\end{itemize}

The following RAM sections are referred to by the final loadable image :

\begin{itemize}
\item bss - All global variables including those initialised by the data section
\item rpage - The window region to align a struct inside.
\item rxbuf - The comms receive buffer.
\item txbuf - The comms transmit buffer.
\end{itemize}


Additionally, the code uses a standard stack at the high end of ram (\texttt{0x4000}) which grows down towards the bss global variable section. When execution begins the gcc start up code initialises the stack frame pointer to the location defined using the symbol \_stack. As each function runs it builds from this location downwards allocating the space it requires for local variables and temporary variables etc. When a function ends the stack frame pointer is restored to where it was before the call occurred. Because of this and the lack of a heap to manually dynamicly allocate memory on memory leaks are not possible.




\section{Programmatic Usage}

There are a number of hand designed schemes in practice in the code base for handling the placement and movement of memory in the device during its normal operation. All of them aim for two things : Speed and Correctness.

The first set of memory operations that occur are to copy all of the live tunable data up to RAM from their locations in Flash. This is done by a series of functions that reside in the same paged flash space as the data to be copied. Placing these items close together ensures fast access and no linker warnings. In order to have the address dictionary compile cleanly a set of pointer variables are also initialised in these same functions. Before each set of four 1k blocks is copied up to RAM from its Flash page the RPAGE value is set to the appropriate value for the data being copied. Once these functions have all run a function is called that samples a specified pin to determine which set of tables to use. This function is checked periodically such that if the pin changes state (the switch is thrown) then so does the behaviour of the firmware. This should happen seamlessly as all operations will just start using the new data set.

Because only one page of paged data can be accessible at a time, we needed a scheme for RAM tuning data access that would enable us to see all 24k of paged data when we needed it and yet still be simple, maintainable and largely foolproof. This was found by forcing the caller to specify the ram page of the data to be accessed in the table lookup and interpolation function call. The table lookup and interpolation functions take a value for RPAGE, store the current value, set RPAGE to the new value, read the data out and finally restore the previous RPAGE before returning. This scheme allows us to leave the finer grained data permanently exposed in the page window for simple and easy access to it it with the only restriction being no access from the ISR code without explicitly swapping pages in and out as the main table lookup and interpolation functions do.

Functions are placed into flash pages in related groups such that minimal page swapping needs to occur for the most frequent calls to be made. ISR functions are all placed together in text1 such that it is easy to keep track of their total size and because they must be located in linear space. The code that burns to and reads from flash is located in text such that it can freely swap pages without removing itself from visibility. Very little else needs to be present in the linear space and as such, most other things are placed in paged space so that we always know how much linear space we have left for various things that must be stored there.

When a serial packet requesting or supplying a data block arrives we must find out where to get it from or put it. This is accomplished with a dictionary approach whereby a single ID number is passed in with a pointer to a struct that describes a memory blocks possible parameters. The function looks up the ID using a large case statement and then populates the struct with the corresponding data. Finally the function returns and the caller can go ahead and use the parameters that have been retrieved. If the request is for a block of RAM the RPAGE value is saved and then set to the value retrieved from the dictionary before reading the RAM data into the transmit buffer. Likewise, when the request is for a block of flash a similar sequence of save, set, read, restore is performed. Again, when data is supplied to be written to RAM, the same thing happens. For flash it is slightly different as the flash function takes care of the swapping around and is passed the page values, RPAGE as a read from RAM value and PPAGE as a write to flash value. The same goes for just writing to flash, however this is followed up by copying the fresh flash data up to ram to ensure syncronisation.



\end{document}
